% Created 2019-05-30 Thu 17:11
% Intended LaTeX compiler: pdflatex
\documentclass[11pt]{article}
\usepackage[utf8]{inputenc}
\usepackage[T1]{fontenc}
\usepackage{graphicx}
\usepackage{grffile}
\usepackage{longtable}
\usepackage{wrapfig}
\usepackage{rotating}
\usepackage[normalem]{ulem}
\usepackage{amsmath}
\usepackage{textcomp}
\usepackage{amssymb}
\usepackage{capt-of}
\usepackage{hyperref}
\author{Kevin Keefe}
\date{\today}
\title{basicCh8IOstream}
\hypersetup{
 pdfauthor={Kevin Keefe},
 pdftitle={basicCh8IOstream},
 pdfkeywords={},
 pdfsubject={},
 pdfcreator={Emacs 26.1 (Org mode 9.2.3)}, 
 pdflang={English}}
\begin{document}

\maketitle
\tableofcontents


\section{header stuff:}
\label{sec:orgc2515d4}
\begin{verbatim}

#include <vector>
#include <iostream>
#include <fstream>
#include <string>
#include <sstream>

using namespace std;
\end{verbatim}
\subsection{header descriptions:}
\label{sec:org15c1a19}
\begin{itemize}
\item vector: useful container space for any object you want.
\item iostream: header which defines cin and cout for cpp. stuff you see on the terminal.
\item fstream: header to allow I/O with files. includes ofstream, ifstream, etc..
\item string: cpp container for char[] but doesn't include the extra null
terminator. slightly more straight forward than a char[], in my opinion.
\item sstream is used for the stringstream type
\begin{itemize}
\item this bad boy can read string line, and will break up the entire string line
into words. you can then perform operations on the string word, like turn it
into hex and then into an integer.
\end{itemize}
\end{itemize}

\section{io stream functions}
\label{sec:org7109e99}
\begin{itemize}
\item quick overview of three different istream reading functions.
\item they're all grouped in the next section with slightly better comments..
\end{itemize}

\begin{verbatim}
istream & read_file(istream & is)
{
  string a;
  auto old_state = is.rdstate();
  getline(is, a);

  if(!is.eof()){
    // this will make the function continue to call itself
    cout << a << endl;
    return read_file(is);
  }
  else
    is.setstate(old_state);
  return is;
}

//this function takes a input istream reference and returns a string value
string & read_fill(istream & is){
  string a;
  is >> a;
  return a;
}
\end{verbatim}
\subsection{function descriptions:}
\label{sec:org6ae95fc}
\begin{itemize}
\item variable is can be input stream: it can be an ifstrem or even cin..
\item the return type must be the istream\&, however since we're using the stream and
it doesn't disappear from the scope of this function call.
\end{itemize}
\begin{verbatim}
istream & read_file(istream & is)
{
  string a;
  auto old_state = is.rdstate();
  getline(is, a);

  if(!is.eof()){
    // this will make the function continue to call itself
    cout << a << endl;
    return read_file(is);
  }
  else
    is.setstate(old_state);
  return is;
}
\end{verbatim}

\begin{itemize}
\item probably the most useful kinda of read statement. this will read a single
string from is and return that value.
\end{itemize}
\begin{verbatim}
//this function takes a input istream reference and returns a string value
string & read_fill(istream & is){
  string a;
  is >> a;
  return a;
}
\end{verbatim}

\section{main function:}
\label{sec:org8deccee}
\begin{verbatim}
int main()
{
  ofstream outfile;
  outfile.open("temp.copy"); // where do you want to store your stuff
  ifstream infile("temp.txt"); // declaring the ifstream like this opens the file immediately
  ifstream infile2("temp.txt");

  //create the vector to push_back
  vector<string> sv, sv2;
  string a, b;

  while(!infile.eof()){
    // a becomes the entire line of the file
    getline(infile, a); // getline reads from infile and stores the line into a

    // add the line to the vector we want
    // make sure getline didn't get the end of the file..
    if (!infile.eof()) sv.push_back(a); // add the line to sv
  }
  // need a second loop for the second file
  while(!infile2.eof())
  {
    b = read_fill(infile2, b); // wrote a similar function to getline, but get a word at a time
    // read_fill will read a string at a time instead of a single line at a tine..
    sv2.push_back(b);
  }
  // at this point, a is a vector a strings which each element being a line in temp.txt
  // similarly for b, it's a vector with each element being a string from temp.txt:w

  // for every element in the string filled by lines vector we just made, send it to standard output
  // note when a variable is passed by reference like this, we can change it value if we want.
  for(string& c : sv){
    // this will print every line
    cout << c << endl;
  }

  // for every element in the string vector we just made, send it to standard output
  for(string& c : sv2){
    // this will print every word
    cout << c << endl;
  }

  // this will turn the string word into a hex value and then into an integer, which is stored in value..
  const string word("ABCDE");
  stringstream ss;
  ss << word;
  int value;
  ss >> std::hex >> value;
  // print the value or store the value wherever you want
  cout << value;
  //remember to close the files
  infile.close();
  outfile.close();

  return 0;
}
\end{verbatim}

\section{example text statement:}
\label{sec:org9b1e0ce}
\begin{itemize}
\item this is one line
\item this is two line
\item this is three line
\item this is four line
\end{itemize}

\subsection{example output}
\label{sec:org269fe74}
each bullet represents a single line of output.
\subsubsection{sv1}
\label{sec:org21d9582}
\begin{itemize}
\item this is one line
\item this is two line
\item this is three line
\item this is four line
\end{itemize}
\subsubsection{sv2}
\label{sec:orge9a0130}
\begin{itemize}
\item is
\item one
\item line
\item this
\item is
\item two
\item line
\item this
\item is
\item three
\item line
\item this
\item is
\item four
\item line
\item line
\end{itemize}
\end{document}
